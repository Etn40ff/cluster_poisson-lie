%\documentclass[12pt,a4paper]{amsart}
\documentclass[a4paper]{amsart}
% this is here to force arXiv to produce a nice output
\pdfoutput=1
\usepackage{mathtools}
\usepackage{amsmath}
\usepackage{amsthm}
\usepackage{amssymb}
\usepackage{amsbsy}
\usepackage{amstext}
\usepackage{amsopn}
\usepackage{enumerate}
\usepackage{xcolor}
\usepackage{graphicx}
\usepackage{microtype}
\usepackage[margin=1in,marginparwidth=0.8in, marginparsep=0.1in]{geometry}
\renewcommand{\baselinestretch}{1.2} % changes page formatting
\usepackage[pagebackref, bookmarks=true, bookmarksopen=true, bookmarksdepth=3,bookmarksopenlevel=2, colorlinks=true, linkcolor=blue, citecolor=blue, filecolor=blue, menucolor=blue, urlcolor=blue]{hyperref}
\usepackage{times} % changes font appearance
\usepackage{stmaryrd}
\usepackage{accents}
\usepackage{bbm}
\usepackage{tikz}
\usepackage[utf8]{inputenc}
\usepackage{cleveref}
%\usepackage{amsfonts}
%\usepackage[]{graphicx}
\usepackage{cleveref}

% define ambients
%\numberwithin{equation}{section}
%\numberwithin{figure}{section}
%\numberwithin{table}{section}
\newtheorem{theorem}{Theorem}%[section]
\newtheorem{proposition}[theorem]{Proposition}
\newtheorem{conjecture}[theorem]{Conjecture}
\newtheorem{corollary}[theorem]{Corollary}
\newtheorem{lemma}[theorem]{Lemma}
\newtheorem{claim}[theorem]{Claim}
\theoremstyle{definition}
\newtheorem{recipe}[theorem]{Recipe}
\newtheorem{remark}[theorem]{Remark}
\newtheorem{example}[theorem]{Example}
\newtheorem{definition}[theorem]{Definition}
\newtheorem{prob}[theorem]{Problem}
\newtheorem{alg}[theorem]{Algorithm}
\newtheorem{ques}[theorem]{Question}

% shothands
\newcommand{\bbC}{\mathbb{C}}
\newcommand{\bbP}{\mathbb{P}}
\newcommand{\bbZ}{\mathbb{Z}}
\newcommand{\bbF}{\mathbb{F}}
\newcommand{\cA}{\mathcal{A}}
%\newcommand{\bx}{\mathbf{x}}
\newcommand{\bx}{X}
\newcommand{\bg}{\mathbf{g}}
\newcommand{\bc}{\mathbf{c}}
\newcommand{\ba}{\mathbf{a}}
\newcommand{\tbx}{\widetilde{\bx}}
\newcommand{\by}{\mathbf{y}}
\newcommand{\bd}{\mathbf{d}}
\newcommand{\tQ}{\widetilde{Q}}
\newcommand{\prQ}{Q_{pr}}
\newcommand{\dpQ}{Q_{dp}}
\newcommand{\mfh}{\mathfrak{h}}
\newcommand{\mfg}{\mathfrak{g}}
\newcommand{\mcO}{\mathcal{O}}
\newcommand{\mcF}{\mathcal{F}}
\newcommand{\mcG}{\mathcal{G}}
\newcommand{\mcH}{\mathcal{H}}

% Commands for marginal notes below
\usepackage[draft]{say}
\newcommand{\sayS}[1]{\say[S]{#1}}

% For setting off new terms
\newcommand{\newword}[1]{\textbf{\emph{#1}}}

\begin{document}
\title{Notes}
\author[Salvatore Stella]{Salvatore Stella}
\address[Salvatore Stella]{University of Haifa}
\email{stella@math.haifa.ac.il}
\date{\today}
\maketitle

\section{Conventions}
It appears that \cite{CP95} and \cite{KS98} use opposite conventions to define brackets, we follow \cite{CP95}.

Let $\Phi$ be a finite type root system and $\Gamma$ the set of simple roots.
Denote by $\mcG$ the associated simple complex Lie group and by $\mfg$ its Lie algebra.

Let $\gamma:\Gamma_1\subsetneq\Gamma\rightarrow\Gamma_2\subsetneq\Gamma$ be a Belavin-Drinfeld map and extend it to a map $\gamma:\Phi_1\subsetneq\Phi\rightarrow\Phi_2\subsetneq\Phi$.
For~$\alpha,\beta\in\Phi$, write~$\alpha\prec_\gamma\beta$ if and only if $\gamma^k(\alpha)=\beta$ and $ht(\alpha)=ht(\beta)$.

Let $\Xi:=\{\xi_\alpha\}_{\alpha\in\Phi}\cup\{\xi_{\alpha^\vee}\}_{\alpha\in\Gamma}$ be the standard basis of $\mfg$.
Denote by $\langle\cdot,\cdot\rangle:\mfg\times\mfg\rightarrow\bbC$ the Killing form and let~$\widehat\Xi:=\{\hat\xi\, |\, \xi \in \Xi\}$ be the basis of $\mfg$ dual to $\Xi$ with respect to the Killing form.
In particular we have $\hat\xi_\alpha=\xi_{-\alpha}$ for all~$\alpha\in\Phi$.
\sayS{This is important in the definition of $r$ otherwise we would need to pick up coefficients.}

Let $t := \sum_{\xi \in \Xi} \xi\otimes \hat\xi$ be the Casimir element in $\mfg\otimes\mfg$ and $t_0 := \sum_{\alpha\in\Gamma} \xi_{\alpha^\vee}\otimes\hat\xi_{\alpha^\vee}$ its restriction to $\mfh\otimes\mfh$. 

Let $s_\gamma\in \mfh\wedge\mfh$ be a solution of the system of equations
\begin{equation}
  \label{eq: system s}
  2\big( (1-\gamma)\alpha\otimes 1\big) s_\gamma
  = 
  \big((1+\gamma)\alpha\otimes 1\big)t_0
  \qquad
  \forall \alpha\in\Gamma_1
\end{equation}
Such a solution is unique up to an element of $\mfh_\gamma\wedge\mfh_\gamma$ where $\mfh_\gamma=\{h\in\mfh\,|\, \alpha(h) = \beta(h)\, \forall \alpha\prec_\gamma\beta\}$.
\sayS{$\mfh_T$ has been renamed to $\mfh_\gamma$}

For a chosen $s_\gamma$ write  
\begin{equation}
  \label{eq: r}
  \sum_{\xi,\zeta\in\Xi} r_{\xi\zeta} \, \xi \otimes \zeta
  = 
  \frac{1}{2}t_0 + s_\gamma + \sum_{\alpha\in\Phi^+} \xi_{-\alpha}\otimes\xi_\alpha + \sum_{\substack{\alpha\prec_\gamma\beta\\ \alpha\in\Phi^+}} \xi_{-\alpha}\wedge \xi_\beta
\end{equation}
\sayS{$r$ depends on $s_\gamma$ but I have no nice way to write this down in formulas. We should get away with it because this is only used here.}
and define the corresponding Poisson bracket of $f,g\in\mcO(\mcG)$ by
\begin{equation}
  \label{eq: bracket}
  \{f,g\}_{\gamma,s_\gamma} 
  := 
  \sum_{\xi,\zeta\in\Xi} r_{\xi\zeta} \big( \xi^Lf\zeta^Lg-\xi^Rf\zeta^Rg \big)
\end{equation}
where, for $\xi\in\mfg$ and $f\in\mcO(\mcG)$, we write 
\[
  \xi^Lf(x):=\left.\frac{d}{dt}f\Big(x \exp(t\xi)\Big)\right|_{t=0}
  \qquad\text{and}\qquad
  \xi^Rf(x):=\left.\frac{d}{dt}f\Big(\exp(t\xi)x\Big)\right|_{t=0}
\]
for the action of the left and right invariant vector fields associated to $\xi$.

\sayS{TODO: rewrite the formula for the bracket in terms of Killing form and Lie algebra gradients?}

For two weights $\lambda,\mu$ in the same $W$-orbit, i.e such that $\lambda=v\omega_i$ and $\mu=u\omega_i$ for some fundamental weight $\omega_i$ and two elements $u,v\in W$, denote by $\Delta_\lambda^\mu$ the generalized minor whose value on $x\in G$ is $[u^{-1}xv]_0^{\omega_i}$.
\section{comparing brackets}
If $\gamma$ extends $\gamma'$ (i.e. $\gamma(\alpha) = \gamma'(\alpha)$ for all $\alpha\in \Gamma'_1$) then $s_\gamma-s_{\gamma'} \in \mfh_{\gamma'}\wedge\mfh_{\gamma'}$ because we are only imposing extra conditions. 
Therefore the bracket $\{\cdot,\cdot\}_{\gamma',s_\gamma}$ is well defined.
In particular $\{\cdot,\cdot\}_{0,s_\gamma}$ is defined for every Belavin-Drinfeld map $\gamma$.

\begin{lemma}
  Suppose $u,u',v,v'\in W$ are such that $\ell(uu') = \ell(u)+\ell(u')$ and $\ell(vv')=\ell(v)+\ell(v')$. 
  Then, for any pair of fundamental weights $\omega_i$ and $\omega_j$,
  \sayS{Probably we do not need this; in case we do think about how to properly write the last two terms; as they are now they make no sense!}
  \sayS{It would be better to write this lemma in terms of $\lambda,\mu,\nu,\eta$ but then I am not sure of the best way to specify the sign. Can we use the partial order on weights?}
  \[
    \frac{\{\Delta_{u\omega_i}^{vv'\omega_i},\Delta_{uu'\omega_j}^{v\omega_j}\}_{0,s_\gamma}}{\Delta_{u\omega_i}^{vv'\omega_i}\Delta_{uu'\omega_j}^{v\omega_j}}
    =
    \frac{\langle \omega_i, u' \omega_j\rangle - \langle v'\omega_i,\omega_j\rangle}{2} + (u\omega_i) \cdot s_\gamma \cdot (u u' \omega_j) - (vv'\omega_i)\cdot s_\gamma \cdot (v\omega_j)
  \]
\end{lemma}
\begin{proof}
  This is \cite[Proposition 4.19]{GSV10} together with a little parsing of the role played by $s_\gamma$.
\end{proof}

Set 
\[
  \mcO^\gamma_+:=\Big\{f\in\mcO(\mcG)\,\Big|\, \xi_{-\alpha}^Lf=0 \,\forall \alpha\in\Phi^+_1\,\text{and}\, \xi_\beta^Rf=0 \,\forall\beta\in\Phi^+_2\Big\}
\]
and
\[
  \mcO^\gamma_-:=\Big\{f\in\mcO(\mcG)\,\Big|\, \xi_{-\alpha}^Rf=0 \,\forall \alpha\in\Phi^+_1\,\text{and}\, \xi_\beta^Lf=0 \,\forall\beta\in\Phi^+_2\Big\}.
\]

\begin{lemma}
  For every $f,g$ in $\mcO^\gamma_+$ (or in $\mcO^\gamma_-$) we have $\{f,g\}_{\gamma,s_\gamma} =\{f,g\}_{0,s_\gamma}$.
\end{lemma}

\begin{proof}
  Unraveling the definition of the Poisson brackets one gets:
  \[
    \{f,g\}_{\gamma,s_\gamma} - \{f,g\}_{0,s_\gamma}
    = 
    \sum_{\substack{\alpha\prec_\gamma\beta\\ \alpha\in\Phi^+}} \Big(\xi_{-\alpha}^L f \xi_\beta^L g - \xi_{-\alpha}^R f \xi_\beta^R g - \xi_\beta^L f \xi_{-\alpha}^L g + \xi_\beta^R f \xi_{-\alpha}^R g\Big)
  \]
  and the claim follows.
\end{proof}

\begin{lemma} For $\alpha\in\Phi$ we have $\xi_\alpha^L \Delta_\lambda^\mu = 0$ if $\alpha^\vee(\mu)\geq0$ and $\xi_\alpha^R \Delta_\lambda^\mu = 0$ if $\alpha^\vee(\lambda)\leq0$.
  \sayS{Check that we really need coroots here and in the other occurrences. I think we do because we want coordinates in the basis of fundamental weights.}
\end{lemma}
\begin{proof}
  It follows immediately from \cite[Lemma 2.6]{RSW17} and the definition of invariant vector fields.
\end{proof}

\begin{corollary}
  \label{cor: good}
  For two weights $\lambda$ and $\mu$ in the same $W$-orbit, if $\alpha^\vee(\lambda)\geq0$ for all $\alpha\in\Phi^+_1$ and $\beta^\vee(\mu)\geq0$ for all~$\beta\in\Phi^+_2$ then $\Delta_\lambda^\mu\in\mcO^\gamma_+$.
  Similarly if $\alpha^\vee(\mu)\leq0$ for all $\alpha\in\Phi^+_1$ and $\beta^\vee(\mu)\leq0$ for all $\beta\in\Phi^+_2$ then $\Delta_\lambda^\mu\in\mcO^\gamma_-$.
  In particular the set $\mcO^\gamma_+$ (resp. $\mcO^\gamma_-$) contains all the leading (resp. trailing) principal minors $\Delta_{\omega_i}^{\omega_i}$ (resp. $\Delta_{-\omega_i}^{-\omega_i}$).
\end{corollary}

\section{Fixing functions}
Fix a Belavin-Drinfeld map $\gamma$.

\begin{definition}
  \sayS{For historical reasons from here onwards we work with $\mcO^\gamma_-$ and trailing minors.  It should be easy enough to reverse all conventions and work with $\mcO^\gamma_+$ and leading minors. This is something one might consider doing before posting these notes.}
  A generalized minor $\Delta_\lambda^\mu$ is \newword{good} if it belongs to $\mcO^\gamma_-$. 
  In view of \cref{cor: good} this means that $\alpha^\vee(\mu)\leq0$ for all~$\alpha\in\Phi^+_1$ and $\beta^\vee(\mu)\leq0$ for all $\beta\in\Phi^+_2$.
\end{definition}
Formally the constant function is a good minor, indeed $1=\Delta_0^0$.

\begin{definition}
  A generalized minor $\Delta_\lambda^\mu$ is \newword{almost good} if either $\lambda=-\omega_i$ for some $i$ and there is a unique simple root $\alpha\in\Gamma_1$ such that $\alpha^\vee(\mu)>0$, or $\mu=-\omega_i$ for some $i$ and there is a unique simple root $\beta\in\Gamma_2$ such that~$\beta^\vee(\mu)>0$. 
\end{definition}

Given a good or almost good generalized minor $\Delta_\lambda^\mu$ define
\[
  \varphi(\Delta_\lambda^\mu) :=
  \begin{cases}
    1 & \mbox{if } \Delta_\lambda^\mu \in \mcO^\gamma_-\\
    \Delta_{\omega_{i^*}}^{-\omega_i} & \mbox{if } \alpha^\vee_j(\mu) > 0 \mbox{ and } \gamma(\alpha_j) = \alpha_i\\
    \Delta_{-\omega_{i}}^{\omega_{i^*}} & \mbox{if } \alpha^\vee_j(\lambda) > 0 \mbox{ and } \gamma(\alpha_i) = \alpha_j\\
  \end{cases}
\]
\sayS{WARNING: maybe here we need to restrict to the case $=1$ rather than $>0$. For $SL_n$ this is irrelevant but it might be important for other types. This could be subsumed in the definition of almost good minor}
where $\omega_{i^*} = -w_0\omega_{i}$.

\begin{lemma}
  If $\gamma$ is \newword{aperiodic} in the sense of \cite{GSV19} then $\varphi^k(\Delta_\lambda^\mu)=1$ for any almost good generalized minor and a sufficiently large $k$.
\end{lemma}
\begin{proof}
  This is basically the definition of being aperiodic.
\end{proof}

From now onward we only talk about aperiodic Belavin-Drinfeld maps.
\sayS{TODO: lift this limitation.}

\begin{definition} 
  Given a good or almost good generalized minor $\Delta_\lambda^\mu$ the associated \newword{core monomial} is the product
  \[
    M_\lambda^\mu := \prod_{k=0}^\infty \varphi^k(\Delta_\lambda^\mu).
  \]
  Since $\gamma$ is aperiodic this is a finite product.
\end{definition}

\begin{definition}
  Given a good or almost good generalized minor $\Delta_\lambda^\mu$ the associated \newword{Belavin-Drinfeld function} is the element
  \[
    \Sigma_\lambda^\mu := \sum_{\zeta\in Z} (-1)^{|\zeta|} \prod_{k=0}^\infty \zeta_k\left(\varphi^k(\Delta_\lambda^\mu)\right)
  \]
  where the summation index runs over the set $Z$ of all sequences of operators $\zeta_k$ of the form
  \[
    \zeta_k = \prod_{\alpha\in A_k}^\rightarrow  \xi^L_\alpha \prod_{\beta\in B_k}^\rightarrow \xi^R_\beta
  \] 
  such that both $A_k$ and $B_k$ are ordered multisubsets of $-\Gamma_1$ or of $\Gamma_2$ subject to the following requirements:
  \begin{itemize}
    \item 
      $A_{-1} = \emptyset = B_{-1} $
    \item
      if $A_k$ and $B_k$ are a multisubset of $-\Gamma_1$ then $A_{k+1} = \gamma(-A_k)$ and $B_{k-1} = \gamma(-B_k)$
    \item
      if $A_k$ and $B_k$ are a multisubset of $\Gamma_2$ then $A_k = \gamma(-A_{k-1})$ and $B_k = \gamma(-B_{k+1})$.
  \end{itemize}
  \sayS{TODO: it would be much nicer and natural here to replace the ordered multisubsets with roots in $\Phi^-_1$ and $\Phi^+_2$. I had the impression that this was not working but one should check again because it was a long time ago at the beginning of the story. One issue could be related to signs of operators correspondign to higher roots but maybe this can be addressed.}
  We used the shorthand $|\zeta|$ to denote the number $\frac{1}{2}\sum_{k=0}^\infty (|A_k|+|B_k|)$.
  The sum is finite because all $\alpha$-strings involved are of finite length.
\end{definition}

\begin{remark}
  Belavin-Drinfeld functions are left and right homogeneous with respect to the action of $\mcH_\gamma:=\exp\mfh_\gamma$.
  This is the case because all operators $\zeta\in Z$ preserve this grading by design.
  In particular Belavin-Drinfeld functions satisfy automatically part of the big conjecture.
\end{remark}

\begin{proposition}
  The functions appearing in \cite{GSV19} are Belavin-Drinfeld functions.
\end{proposition}
\begin{proof}
  This basically follows from the definition of the operators $\zeta$ together with the block structure of those functions.
  \say{TODO: write a formal proof in case we need to.}
\end{proof}

\begin{conjecture}
  If $\Delta_\lambda^\mu$ and $\Delta_\nu^\eta$ are log-canonical with respect to $\{\cdot,\cdot\}_{0,s_\gamma}$ then
  \[
    \frac{\{\Sigma_\lambda^\mu,\Sigma_\nu^\eta\}_{\gamma,s_\gamma}}{\Sigma_\lambda^\mu \Sigma_\nu^\eta} 
    = 
    \frac{\{M_\lambda^\mu,M_\nu^\eta\}_{0,s_\gamma}}{M_\lambda^\mu M_\nu^\eta}.
  \]
\end{conjecture}

We collect here some observations that might be relevant in proving the conjecture.
\begin{itemize}
  \item
   For any almost good generalized minor $\Delta_\lambda^\mu$ the generalized minor $1\neq\varphi^k(\Delta_\lambda^\mu)$ for $k>0$ is log-canonical with any other generalized minor with respect to the Poisson bracket $\{\cdot,\cdot\}_{0,s_\gamma}$.
   Indeed it is a frozen variable in the classical cluster structure.
   In particular this, together with the assumption in the conjecture, implies that $\Sigma_\lambda^\mu$ and $\Sigma_\nu^\eta$ are log-canonical with respect to $\{\cdot,\cdot\}_{\gamma,s_\gamma}$!

 \item
   One can replace the right hand side in the conjecture by the equivalent
   \[
     \frac{\{M_\lambda^\mu,M_\nu^\eta\}_{0,s_\gamma}}{M_\lambda^\mu M_\nu^\eta}
     =
     \sum_{k=0}^\infty\sum_{\ell=0}^\infty \frac{\{\varphi^k(\Delta_\lambda^\mu),\varphi^\ell(\Delta_\nu^\eta)\}_{0,s_\gamma}}{\varphi^k(\Delta_\lambda^\mu)\varphi^\ell(\Delta_\nu^\eta)}
   \]
   this could be more useful to use in an induction on $min(\{k | \varphi^k(\Delta_\lambda^\mu)=1\}\cup \{\ell | \varphi^\ell(\Delta_\nu^\eta)=1\} )$ since one can ``peel off'' one layer of minors at a time as follows.
   Write $\Delta_\rho^\sigma = \varphi(\Delta_\lambda^\mu)$ then the conjecture is equivalent to
   \[
     \frac{\{\Sigma_\lambda^\mu,\Sigma_\nu^\eta\}_{\gamma,s_\gamma}}{\Sigma_\lambda^\mu \Sigma_\nu^\eta}
     -
     \frac{\{\Sigma_\rho^\sigma,\Sigma_\nu^\eta\}_{\gamma,s_\gamma}}{\Sigma_\rho^\sigma \Sigma_\nu^\eta}
     =
     \sum_{\ell=0}^\infty \frac{\{\Delta_\lambda^\mu,\varphi^\ell(\Delta_\nu^\eta)\}_{0,s_\gamma}}{\Delta_\lambda^\mu\varphi^\ell(\Delta_\nu^\eta)}
     .
   \]

 \item
   Looking ahead, one should be able to read directly the exchange matrix of Belavin-Drinfeld functions with respect to $\{\cdot,\cdot\}_{\gamma,s_\gamma}$ out of the monomial exchange matrix of the corresponding core monomials with respect to $\{\cdot,\cdot\}_{0,s_\gamma}$.
   This is the computation Misha Shapiro was doing during Summer 2018 and of which I did not fully understand the implications at the time.
\end{itemize}

\section{Clusters}

\begin{proposition}
  Fix a Belavin-Drinfeld map $\gamma$. 
  Any Berenstein-Fomin-Zelevinsky seed contains at least $|\Phi_1|$ generalized minors that are not good.
\end{proposition}

\begin{proposition}
  For any Belavin-Drinfeld map $\gamma$ there exists a double reduced word such that the corresponding Berenstein-Fomin-Zelevinsky seed consists only of good and almost good generalized minors.
\end{proposition}


\section{Words}
\begin{proposition}
  For any Belavin-Drinfeld map $\gamma$ there exists a double reduced word such that the corresponding BFZ seed consists only of good and almost good minors
\end{proposition}


For any double reduced word $\bf{i}$ of $w_0$, denote by $\operatorname{BFZ}(\bf{i})$ the set of minors in the seed associated to $\bf{i}$ in \cite{BFZ05}.

\begin{corollary}
  For any double reduced word $\bf{i}$ of $w_0$, both  $\operatorname{BFZ}(\bf{i})\cap\mathcal{D}^\gamma_+$ and $\operatorname{BFZ}(\bf{i})\cap\mathcal{D}^\gamma_-$ are sets of log-canonical functions with respect to the bracket $\{\cdot,\cdot\}_\gamma$.
  Moreover the matrices $\Omega_\gamma$ and $\Omega_0$ restricted to these subsets coincide.
\end{corollary}

\begin{proposition}
  Write $w_0(\Gamma_i)$ for the longest element of the Dynkin diagram $\Gamma_i$.
  We abuse the notation and write also $w_0=w_0(\Gamma_i)u$ for any reduced expression of $\omega_0$ that begins by any reduced expression of $w_0(\Gamma_i)$.

  The set
  \[
    \operatorname{BFZ}\big(w_0(\Gamma_1),u,-v,-w_0(\Gamma_2)\big)\cap\mathcal{D}^\gamma_+
  \]
  has the maximum cardinality among the sets of log-canonical functions $\operatorname{BFZ}(\bf{i})\cap\mathcal{D}^\gamma_+$ as $\bf{i}$ varies; its cardinality is precisely $n+|\Phi|-|\Phi_{\Gamma_1}|$. 
  Moreover it contains all the leading principal minors.

  Similarly the set
  \[
    \operatorname{BFZ}\big(-w_0(\Gamma_2),-v,u,w_0(\Gamma_1)\big)\cap\mathcal{D}^\gamma_-
  \]
  has the maximum cardinality among the sets of log-canonical functions $\operatorname{BFZ}(\bf{i})\cap\mathcal{D}^\gamma_-$ as $\bf{i}$ varies; its cardinality is precisely $n+|\Phi|-|\Phi_{\Gamma_1}|$. 
  Moreover it contains all the trailing principal minors.
\end{proposition}
\begin{proof}
  This is proved in some other notes but I did not have time to copy it over yet.
  The proof is lengthy but uses just the definitions and some simple reasoning on the pairing of roots and weights.
\end{proof}

\begin{remark}
  The shape of the reduced words in the previous proposition guarantees that the set $\operatorname{BFZ}\big(w_0(\Gamma_1),u,-v,-w_0(\Gamma_2)\big)$ only contains minors of the forms $\Delta_{\omega_i}^\mu$ and $\Delta_\lambda^{\omega_i}$.
  Similarly the set $\operatorname{BFZ}\big(-w_0(\Gamma_2),-v,u,w_0(\Gamma_1)\big)$ only contains minors of the forms $\Delta_{-\omega_i}^\mu$ and $\Delta_\lambda^{-\omega_i}$.
\end{remark}

\section{Clusters}
For this section we only consider double reduced words of the form $\big(-w_0(\Gamma_2),-v,u,w_0(\Gamma_1)\big)$.
We could use the other with leading minors, which would probably be more natural, but we stick to this for the moment for historical reasons.
Eventually the goal is to get completely rid of this assumption.
Note that in a couple of cases I did test the following recipes without this assumptions and things seemed to work, there were just more functions to adjust.

For the moment we assume also that the BD-triple has no cycles.
I still haven't thought of how to fix the following to deal with that case.
Note that this assumption basically forces the group to be $SL_{n+1}$.
Indeed there are also few small cases in type $D_n$ but I did not test those because my code is not clean enough yet.

Finally we also assume that the double reduced word is such that for any minor $\Delta_{-\omega_i}^\mu$ (resp. $\Delta_{\lambda}^{-\omega_i}$) in 
\[
  \operatorname{BFZ}\big(-w_0(\Gamma_2),-v,u,w_0(\Gamma_1)\big)\setminus \mathcal{D}^\gamma_-
\]
there is a unique simple root $\alpha_i\in\Delta_{\Gamma_1}$ (resp. $\alpha_i\in\Delta_{\Gamma_2}$) such that $\alpha_i^\vee(\mu)>0$ (resp. $\alpha_i^\vee(\lambda)>0$).
This assumption is easy to guarantee by picking reduced expressions for $w_0(\Gamma_i)$ that are greedy.
This assumption is completely artificial and should be removed as soon as possible, in some examples we have a recipe to get rid of it but somehow this does not work in general.

\begin{recipe}
  Pick a function in $\operatorname{BFZ}\big(-w_0(\Gamma_2),-v,u,w_0(\Gamma_1)\big)\setminus \mathcal{D}^\gamma_-$. 
  To fix ideas say $\Delta_{-\omega_i}^\mu$.
  Let $j$ be the unique (by hypothesis) index such that $\alpha_j^\vee(\mu)>0$.
  Construct the \newword{$\gamma$-leading monomial}
  \[
    \Delta_{-\omega_i}^\mu 
    \Delta_{\omega_{\gamma(j)^*}}^{-\omega_{\gamma(j)}}
    \Delta_{-\omega_{\gamma^{-1}(\gamma(j)^*)}}^{\omega_{\gamma^{-1}(\gamma(j)^*)^*}}
    \cdots
  \]
  More precisely: if a minor is ``bad'' in the columns (i.e. the superscript $\mu$ has  $\alpha_k^\vee(\mu)>0$ for some $k\in\Gamma_1$) then multiply on the right by $\Delta_{\omega_{\gamma(k)^*}}^{-\omega_{\gamma(k)}}$.
  It it is ``bad'' in the rows (i.e. the subscript $\lambda$ has  $\alpha_k^\vee(\lambda)>0$ for some $k\in\Gamma_2$) then multiply on the right by $\Delta_{-\omega_{\gamma^{-1}(k)}}^{\omega_{\gamma^{-1}(k)^*}}$.
  Repeat as long as the last minor multiplied is ``bad''; the process will end because we assume that the BD-triple has no cycles.
\end{recipe}

By abusing notation we can use the same recipe on a minor in $\operatorname{BFZ}\big(-w_0(\Gamma_2),-v,u,w_0(\Gamma_1)\big)\cap \mathcal{D}^\gamma_-$, this will produce a leading monomial with a unique term.

\begin{recipe}
  Consider the collection of operators 
  \[
    \operatorname{D_\gamma}:=\Big\{ 
      F_{\alpha_i}^L\otimes E_{\alpha_j}^L, 
      F_{\alpha_i}^R\otimes E_{\alpha_j}^R, 
      E_{\alpha_j}^L\otimes F_{\alpha_i}^L, 
      E_{\alpha_j}^R\otimes F_{\alpha_i}^R
      \,\big|\,\alpha_i\prec\alpha_j
    \Big\}.
  \]
  Given a minor $\Delta_\lambda^\mu$ in $\operatorname{BFZ}\big(-w_0(\Gamma_2),-v,u,w_0(\Gamma_1)\big)$, let the associated \newword{$\gamma$-canonicalized function} $\Sigma_\lambda^\mu$  be the function obtained summing, with signs, all the monomials constructed from the associated $\gamma$-leading monomial by applying a sequence of operators from $\operatorname{D_\gamma}$ (including the empty sequence) to each adjacent pair of terms.
  The sign is given by the total number of operators applied.
  The sum is taken without repetitions: if two sequences produce the same monomial we only take it once.
\end{recipe}

Note that, if a minor $\Delta_\lambda^\mu$ is in $\operatorname{BFZ}\big(-w_0(\Gamma_2),-v,u,w_0(\Gamma_1)\big)\cap \mathcal{D}^\gamma_-$, then $\Delta_\lambda^\mu=\Sigma_\lambda^\mu$.

\begin{claim}
  The collection
  \[
    \Big\{
      \Sigma_\lambda^\mu\,\big|\, \Delta_\lambda^\mu\in \operatorname{BFZ}\big(-w_0(\Gamma_2),-v,u,w_0(\Gamma_1)\big)
    \Big\}
  \]
  consists of $n+|\Phi|$ algebraically independent functions that are pairwise log-canonical with respect to $\{\cdot,\cdot\}_\gamma$.
\end{claim}

\begin{remark}
  The functions $\Sigma_\lambda^\mu$ are homogeneous with respect to the two-sided action of the restricted Cartan.
  The reason is that each operator in $\operatorname{D_\gamma}$ has degree 0.
  In particular part of the big conjecture is automatic.
\end{remark}

\section{Examples}
\begin{example}
  Type $A_2$, $\Gamma_1=\{1\}$, $\Gamma_2=\{2\}$, $\gamma(1)=2$.
  There is only one double reduced word satisfying the above conditions: ${\bf i}=(-2, -1, -2, 1, 2, 1)$.
  The corresponding set $\operatorname{BFZ}\big(-2, -1, -2, 1, 2, 1\big)$ is 
  \[
    \left\{
    \Delta_{\omega_{2}}^{-\omega_{1}},
    \Delta_{\omega_{1}}^{-\omega_{2}},
    \Delta_{\omega_{1} - \omega_{2}}^{-\omega_{1}},
    \Delta_{-\omega_{2}}^{-\omega_{2}},
    \Delta_{-\omega_{1}}^{-\omega_{1}}, 
    \Delta_{-\omega_{2}}^{-\omega_{1} + \omega_{2}},
    \Delta_{-\omega_{1}}^{\omega_{2}},
    \Delta_{-\omega_{2}}^{\omega_{1}}
    \right\}
  \]
  There are only two ``bad'' functions: $\Delta_{\omega_{2}}^{-\omega_{1}}$ and $\Delta_{-\omega_{2}}^{\omega_{1}}$.
  The associated $\gamma$-canonicalized functions are
  \[
    \Sigma_{\omega_{2}}^{-\omega_{1}}=
    \Delta_{\omega_{2}}^{-\omega_{1}}\Delta_{-\omega_1}^{\omega_2} - E_{\alpha_2}^R\Delta_{\omega_{2}}^{-\omega_{1}} F_{\alpha_1}^R\Delta_{-\omega_1}^{\omega_2}
  \]
  and
  \[
    \Sigma_{-\omega_{2}}^{\omega_{1}}=
    \Delta_{-\omega_{2}}^{\omega_{1}}\Delta_{\omega_1}^{-\omega_2} - F_{\alpha_1}^L\Delta_{-\omega_{2}}^{\omega_{1}}E_{\alpha_2}^L\Delta_{\omega_1}^{-\omega_2}
  \]
  This example is 1 mutation away from the one in the corresponding GSV construction: it suffices to mutate at $\Delta_{-\omega_{2}}^{-\omega_{1} + \omega_{2}}$ to recover it.
  Note that there might be a mismatch in conventions so that we might have to consider $\gamma^{-1}$ to make things match.
\end{example}

\begin{example}
  Type $A_3$, $\Gamma_1=\{1,2\}$, $\Gamma_2=\{2,3\}$, $\gamma(1)=2, \gamma(2)=3$,\par${\bf i}=(-2, -3, -2, -1, -2, -3, 1, 2, 3, 2, 1, 2)$.
  This is the list of all $\gamma$-leading monomials associated to ${\bf i}$ in this case
  \begin{align*}
    \Big\{
    \Delta_{ \omega_{3} }^{ -\omega_{1} } \Delta_{ -\omega_{2} }^{ \omega_{2} } \Delta_{ \omega_{1} }^{ -\omega_{3} },\,
    \Delta_{ \omega_{2} }^{ -\omega_{2} } \Delta_{ -\omega_{1} }^{ \omega_{3} },\,
    \Delta_{ \omega_{1} }^{ -\omega_{3} },\,
    \Delta_{ \omega_{1} - \omega_{2} + \omega_{3} }^{ -\omega_{2} } \Delta_{ -\omega_{2} }^{ \omega_{2} } \Delta_{ \omega_{1} }^{ -\omega_{3} },\,
    \Delta_{ \omega_{1} - \omega_{2} }^{ -\omega_{1} },\,
    \Delta_{ \omega_{1} - \omega_{3} }^{ -\omega_{2} },\,
    \Delta_{ -\omega_{3} }^{ -\omega_{3} },\,
    \Delta_{ -\omega_{2} }^{ -\omega_{2} },\,\\
    \Delta_{ -\omega_{1} }^{ -\omega_{1} },\,
    \Delta_{ -\omega_{3} }^{ -\omega_{2} + \omega_{3} },\,
    \Delta_{ -\omega_{2} }^{ -\omega_{1} + \omega_{3} },\,
    \Delta_{ -\omega_{1} }^{ \omega_{3} },\,
    \Delta_{ -\omega_{2} }^{ \omega_{1} - \omega_{2} + \omega_{3} } \Delta_{ \omega_{2} }^{ -\omega_{2} } \Delta_{ -\omega_{1} }^{ \omega_{3} },\,
    \Delta_{ -\omega_{3} }^{ \omega_{1} } \Delta_{ \omega_{2} }^{ -\omega_{2} } \Delta_{ -\omega_{1} }^{ \omega_{3} },\,
    \Delta_{ -\omega_{2} }^{ \omega_{2} } \Delta_{ \omega_{1} }^{ -\omega_{3} }
    \Big\}
  \end{align*}
  An example of $\gamma$-canonicalized function is
  \begin{align*}
    \Sigma_{ \omega_{3} }^{ -\omega_{1} } 
    = &
    \Delta_{ \omega_{3} }^{ -\omega_{1} } \Delta_{ -\omega_{2} }^{ \omega_{2} } \Delta_{ \omega_{1} }^{ -\omega_{3} }
    -
    E_{\alpha_3}^R\Delta_{ \omega_{3} }^{ -\omega_{1} } F_{\alpha_2}^R\Delta_{ -\omega_{2} }^{ \omega_{2} } \Delta_{ \omega_{1} }^{ -\omega_{3} }
    + 
    E_{\alpha_2}^RE_{\alpha_3}^R\Delta_{ \omega_{3} }^{ -\omega_{1} } F_{\alpha_1}^RF_{\alpha_2}^R\Delta_{ -\omega_{2} }^{ \omega_{2} } \Delta_{ \omega_{1} }^{ -\omega_{3} }\\&
    -
    \Delta_{ \omega_{3} }^{ -\omega_{1} } F_{\alpha_2}^L\Delta_{ -\omega_{2} }^{ \omega_{2} } E_{\alpha_3}^L\Delta_{ \omega_{1} }^{ -\omega_{3} }
    +
    E_{\alpha_3}^R\Delta_{ \omega_{3} }^{ -\omega_{1} } F_{\alpha_2}^R F_{\alpha_2}^L\Delta_{ -\omega_{2} }^{ \omega_{2} } E_{\alpha_3}^L\Delta_{ \omega_{1} }^{ -\omega_{3} }\\&
    - 
    E_{\alpha_2}^RE_{\alpha_3}^R\Delta_{ \omega_{3} }^{ -\omega_{1} } F_{\alpha_1}^RF_{\alpha_2}^R F_{\alpha_2}^L\Delta_{ -\omega_{2} }^{ \omega_{2} } E_{\alpha_3}^L\Delta_{ \omega_{1} }^{ -\omega_{3} }
    +
    \Delta_{ \omega_{3} }^{ -\omega_{1} } F_{\alpha_1}^L F_{\alpha_2}^L\Delta_{ -\omega_{2} }^{ \omega_{2} } E_{\alpha_2}^LE_{\alpha_3}^L\Delta_{ \omega_{1} }^{ -\omega_{3} }\\&
    -
    E_{\alpha_3}^R\Delta_{ \omega_{3} }^{ -\omega_{1} } F_{\alpha_2}^R  F_{\alpha_1}^L F_{\alpha_2}^L\Delta_{ -\omega_{2} }^{ \omega_{2} }E_{\alpha_2}^L E_{\alpha_3}^L\Delta_{ \omega_{1} }^{ -\omega_{3} }
    + 
    E_{\alpha_2}^RE_{\alpha_3}^R\Delta_{ \omega_{3} }^{ -\omega_{1} } F_{\alpha_1}^RF_{\alpha_2}^R  F_{\alpha_1}^LF_{\alpha_2}^L\Delta_{ -\omega_{2} }^{ \omega_{2} }E_{\alpha_2}^L E_{\alpha_3}^L\Delta_{ \omega_{1} }^{ -\omega_{3} }
  \end{align*}
\end{example}
\section{Tricks for computations}
The operators $X^L$ and $Y^R$ commutes.

In the case of $SL_{n+1}$, and in other cases depending on the length of root strings in the appropriate representations
\begin{align*}
  E_{\alpha_i}^R \Delta_\lambda^\mu &= \Delta_{\lambda-\alpha_i}^\mu\\
  F_{\alpha_i}^R \Delta_\lambda^\mu &= \Delta_{\lambda+\alpha_i}^\mu\\
  E_{\alpha_i}^L \Delta_\lambda^\mu &= \Delta_\lambda^{\mu+\alpha_i}\\
  F_{\alpha_i}^L \Delta_\lambda^\mu &= \Delta_\lambda^{\mu-\alpha_i}
\end{align*}
whenever the right hand side make sense (i.e. if the two weights in the resulting minors are in the same $W$-orbit) otherwise the result is $0$.
A similar result holds also for vector fields associated to roots of higher height but there is a sign appearing and I do not have a simple way to explain it.

For $SL_{n+1}$ the vector fields operations above correspond to shifting rows or columns in the minors.
Keeping this in mind it is easy to write GSV functions in the form of $\gamma$-canonicalized functions.

\section{Miscellanea/Questions}
I have no proof of anything written in here from Section 3 onward.

I still have no conjecture on how to read the exchange matrix out of this construction.

The two recipes in section 3 needs to be updated to deal with:
\begin{itemize}
  \item
    Minors bad with respect to more than one simple root.
  \item
    Minors bad both in rows and columns (this is to allow shuffled double reduced words).
  \item
    Cyclic BD-triples (this is needed to complete $SL_{n+1}$ and to deal with the other groups).
\end{itemize}

Given that the functions $\Sigma_\lambda^\mu$ are written in terms of minors, it would be good to have a statement like \cite[Corollary 4.21]{GSV10}.
My impression is that the difficulty is just in the $r_0$ part.
I have to parse Misha's suggestion in order to see if this is really the case.


% bibliography
\bibliographystyle{amsalpha}
\bibliography{bibliography}

\end{document}

%bibliography
%\bibliographystyle{amsalpha}
%\bibliography{bibliography}
\end{document}


